%% bare_jrnl.tex
%% V1.3
%% 2007/01/11
%% by Michael Shell
%% see http://www.michaelshell.org/
%% for current contact information.
%%
%% This is a skeleton file demonstrating the use of IEEEtran.cls
%% (requires IEEEtran.cls version 1.7 or later) with an IEEE journal paper.
%%
%% Support sites:
%% http://www.michaelshell.org/tex/ieeetran/
%% http://www.ctan.org/tex-archive/macros/latex/contrib/IEEEtran/
%% and
%% http://www.ieee.org/

%%*************************************************************************
%% Legal Notice:
%% This code is offered as-is without any warranty either expressed or
%% implied; without even the implied warranty of MERCHANTABILITY or
%% FITNESS FOR A PARTICULAR PURPOSE! 
%% User assumes all risk.
%% In no event shall IEEE or any contributor to this code be liable for
%% any damages or losses, including, but not limited to, incidental,
%% consequential, or any other damages, resulting from the use or misuse
%% of any information contained here.
%%
%% All comments are the opinions of their respective authors and are not
%% necessarily endorsed by the IEEE.
%%
%% This work is distributed under the LaTeX Project Public License (LPPL)
%% ( http://www.latex-project.org/ ) version 1.3, and may be freely used,
%% distributed and modified. A copy of the LPPL, version 1.3, is included
%% in the base LaTeX documentation of all distributions of LaTeX released
%% 2003/12/01 or later.
%% Retain all contribution notices and credits.
%% ** Modified files should be clearly indicated as such, including  **
%% ** renaming them and changing author support contact information. **
%%
%% File list of work: IEEEtran.cls, IEEEtran_HOWTO.pdf, bare_adv.tex,
%%                    bare_conf.tex, bare_jrnl.tex, bare_jrnl_compsoc.tex
%%*************************************************************************

\documentclass[journal]{IEEEtran}

\newcommand{\subparagraph}{}

\usepackage{amsmath,
            amssymb,
            amsthm,
            atbegshi,
            caption,
            subcaption,
            epigraph,
            etoolbox,
            enumitem,
            fancyhdr,
            geometry,
            graphicx,
            hyperref,
            kpfonts,
            lipsum,
            longtable,
            natbib,
            tabulary,
            thmtools,
            tikz,
            tikzpagenodes,
            titletoc,
            titlesec,
            tocloft,
            url,
            wrapfig
}
\usepackage[utf8]{inputenc}


\begin{document}
%
% paper title
% can use linebreaks \\ within to get better formatting as desired
\title{2048 Expectimax Solver}

\author{Anders Sildnes, Andrej Leitner~\IEEEmembership{students of NTNU}% <-this % stops a space. jobtitle in memberkj
}% \thanks{Utsendt 2014}}

% The paper headers
\markboth{2048 Expectimax Solver}%
{h}

% make the title area
\maketitle

\begin{abstract}
    This text answers assignment 5: writing a solver for the game 2048.
    The purpose of the game is to slide tiles with numbers $2^{n}$, and join them
    together to form 2048 ($2^{11}$), or possibly higher. 
    We present a solver using Expectimax algorithm. We explain our choice of
    heuristics and results.
\end{abstract}

% \begin{IEEEkeywords}
%     Stuff
% \end{IEEEkeywords}
\IEEEPARstart{2}{048} can be thought of as a 2-player, turn-based game. The
opponent places tiles valued either $2$ or $4$ in an availible location. Then,
the other player makes a move to slide all tiles in a given direction. This
goes back and forth until there are no more availible moves. 

In a turn-based game, you have the time to consider the consequences for each possible
action. This gives computers immense advantages over humans:
IBM's chess-solving machine ``Deep Blue'' won against Garry Kasparov in 1997 considered more than
200 million possible moves per second\footnote{For more, see
    \href{http://www-03.ibm.com/ibm/history/ibm100/us/en/icons/deepblue/}
{http://www-03.ibm.com/ibm/history/ibm100/us/en/icons/deepblue/}}.

In the case of 2048, this could yield a valid solution in a short amount of
time. However, not everyone has access to such fast hardware and
multi-threading so a way to prune the search space is needed.

\section*{Minimax Algorithm}
The purpose of minimax is to minimize loss for a worst possible scenario.
This means choosing an action such that your opponent can do ``least amount of damages'',
i.e.\ leaving you in a promising state. This is also called \textit{adversial search}.
For 2048 this would mean always making a move such that no matter where the next
tile goes, and no matter what value, you will still find a way for success.

% PERFECT INFORMATION
% PLY

2048, however, is a stochastic game. The opponent will never purposly select a
bad tile, nor \textit{choose} a bad value. This is to say that there is no opponent,
so it is possible to always land a best case scenario. Therefore, we felt that minimax,
considering the worst case, is too pessimistic in its search space.
Furthermore, the distribution of numbered tiles are given:
there is a 90\% chance of getting a tile valued 2, and 10\% chance of getting a tile
valued 4. The location is random. Also note that in some cases, a tile
in position $i,j$  is no different from having a tile in position $i+i,j$ if the
next move it to slide the tiles downward. Thus the possible search space
is not too big and one can use statistics to get accurate enough resuls.


Using stochastic information in adversial search leads to expectimax. 
After expanding the search space up to a given depth, you can back up from each
node and introduce a chance node. The chance nodes multiply the value to each
of its children (given a state, the objective value to each of the children).
The chance nodes are used when pruning the search space. Their importance is that
unlikely values will yield a lower penalty to the overall score of the search node,
such that the node is not necessarily pruned away.

\section*{Heuristic}
To assess whether or not a current game state is good or not, we have different
objective functions to assess game state. Using these to advance in the
solution space is called heuristical search. Our heuristics are developed based
on ideas we found online and through some play of our own.
\subsection*{Gradients}
If a large value is in the middle of the board, it will be hard to pair up
with other adjacent tiles. Therefore, we would like to set a natural preference
for having larger values clustered in a corner. This way, they do not interfere
with other lower-valued tiles. Next to the highest value tile, you want other,
lower valued cells. This extends throughout the board. However, having large values
in two corners implies that we have two (or more) tiles with large values, separated
by lower tiles. Therefore, we penalize boards where large values are far apart.

To calculate the value of a board, we apply a 4x4 mask and retrieve the
sum of adding all the values in the masked board.  
The mask is shown \autoref{fig:gradient}. Each number in the tiles represents a scalar
which is multiplied with the value from the board. Since large value might be clustered
in all corners, we apply 4 masks, one for each corner respectively. Then,
the largest resulting value is chosen as the one to be used as a value for the state.

\begin{figure}[Hb]
\centering
  % TikZ picture with origin upper left
    \begin{tikzpicture}[yscale=-1] 
        % 4x4 grid
        \draw (0, 0) grid (4, 4);
        % origin point
        % \draw [color=blue, fill=blue] (0, 0) circle (0.1);
        % % x-axis
        % \draw [thick,->] (0, 0) -- (3.5, 0);
        % % y-axis
        % \draw [thick,->] (0, 0) -- (0, 3.5);
        % origin label
        % \node at (-0.5, 0.5) {1};
        \node at (0.5, 0.5) {3};
        \node at (1.5, 0.5) {2};
        \node at (2.5, 0.5) {1};
        \node at (3.5, 0.5) {0};

        \node at (0.5, 1.5) {2};
        \node at (1.5, 1.5) {1};
        \node at (2.5, 1.5) {0};
        \node at (3.5, 1.5) {-1};

        \node at (0.5, 2.5) {1};
        \node at (1.5, 2.5) {0};
        \node at (2.5, 2.5) {-1};
        \node at (3.5, 2.5) {-2};

        \node at (0.5, 3.5) {0};
        \node at (1.5, 3.5) {-1};
        \node at (2.5, 3.5) {-2};
        \node at (3.5, 3.5) {-3};
        % % x-axis label
        % \node at (4.5, -0.5) {200px};
        % % y-axis label
        % \node at (0, 5) {200px};
    \end{tikzpicture}
    \caption{Our gradient mask}
\label{fig:gradient}
\end{figure}

\section*{Why not use more heuristics?}
Many heuristic functions exist:
\begin{itemize}
    \item \textbf{Free tiles} \textendash{} giving scores based on how many cells are left.
    \item \textbf{Board score} \textendash{} whenever two cells merge, the board score increases.
    \item \textbf{Non-monotnic rows/coumns} \textendash{} prefer boards where there are large, but similar
        tiles.
    \item Possible merges \textendash{} prefer boards where many merges are possible,
        i.e.\ boards with tiles of the same value.
\end{itemize}


\section*{end}


\end{document}
