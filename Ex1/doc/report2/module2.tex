%% bare_jrnl.tex
%% V1.3
%% 2007/01/11
%% by Michael Shell
%% see http://www.michaelshell.org/
%% for current contact information.
%%
%% This is a skeleton file demonstrating the use of IEEEtran.cls
%% (requires IEEEtran.cls version 1.7 or later) with an IEEE journal paper.
%%
%% Support sites:
%% http://www.michaelshell.org/tex/ieeetran/
%% http://www.ctan.org/tex-archive/macros/latex/contrib/IEEEtran/
%% and
%% http://www.ieee.org/

%%*************************************************************************
%% Legal Notice:
%% This code is offered as-is without any warranty either expressed or
%% implied; without even the implied warranty of MERCHANTABILITY or
%% FITNESS FOR A PARTICULAR PURPOSE! 
%% User assumes all risk.
%% In no event shall IEEE or any contributor to this code be liable for
%% any damages or losses, including, but not limited to, incidental,
%% consequential, or any other damages, resulting from the use or misuse
%% of any information contained here.
%%
%% All comments are the opinions of their respective authors and are not
%% necessarily endorsed by the IEEE.
%%
%% This work is distributed under the LaTeX Project Public License (LPPL)
%% ( http://www.latex-project.org/ ) version 1.3, and may be freely used,
%% distributed and modified. A copy of the LPPL, version 1.3, is included
%% in the base LaTeX documentation of all distributions of LaTeX released
%% 2003/12/01 or later.
%% Retain all contribution notices and credits.
%% ** Modified files should be clearly indicated as such, including  **
%% ** renaming them and changing author support contact information. **
%%
%% File list of work: IEEEtran.cls, IEEEtran_HOWTO.pdf, bare_adv.tex,
%%                    bare_conf.tex, bare_jrnl.tex, bare_jrnl_compsoc.tex
%%*************************************************************************

\documentclass[journal]{IEEEtran}
\newcommand{\subparagraph}{}


\usepackage{amsmath,
            amssymb,
            amsthm,
            atbegshi,
            caption,
            color,
            subcaption,
            epigraph,
            etoolbox,
            enumitem,
            fancyhdr,
            geometry,
            graphicx,
            hyperref,
            kpfonts,
            lipsum,
            listings,
            longtable,
            natbib,
            tabulary,
            thmtools,
            tikz,
            tikzpagenodes,
            titletoc,
            titlesec,
            tocloft,
            url,
            wrapfig
}
\usepackage[utf8]{inputenc}

\definecolor{mygreen}{rgb}{0,0.6,0}
\definecolor{mygray}{rgb}{0.5,0.5,0.5}
\definecolor{mymauve}{rgb}{0.58,0,0.82}

\lstset{ %
  backgroundcolor=\color{white},   % choose the background color; you must add \usepackage{color} or \usepackage{xcolor}
  basicstyle=\footnotesize,        % the size of the fonts that are used for the code
  breakatwhitespace=false,         % sets if automatic breaks should only happen at whitespace
  breaklines=true,                 % sets automatic line breaking
  captionpos=b,                    % sets the caption-position to bottom
  commentstyle=\color{mygreen},    % comment style
  deletekeywords={...},            % if you want to delete keywords from the given language
  escapeinside={\%*}{*)},          % if you want to add LaTeX within your code
  extendedchars=true,              % lets you use non-ASCII characters; for 8-bits encodings only, does not work with UTF-8
  firstnumber=1,
  frame=single,                    % adds a frame around the code
  keepspaces=true,                 % keeps spaces in text, useful for keeping indentation of code (possibly needs columns=flexible)
  keywordstyle=\color{blue},       % keyword style
  language=Python,                 % the language of the code
  morekeywords={*,...},            % if you want to add more keywords to the set
  numberfirstline=true,
  numbers=left,                    % where to put the line-numbers; possible values are (none, left, right)
  numbersep=5pt,                   % how far the line-numbers are from the code
  numberstyle=\tiny\color{mygray}, % the style that is used for the line-numbers
  rulecolor=\color{black},         % if not set, the frame-color may be changed on line-breaks within not-black text (e.g. comments (green here))
  showspaces=false,                % show spaces everywhere adding particular underscores; it overrides 'showstringspaces'
  showstringspaces=false,          % underline spaces within strings only
  showtabs=false,                  % show tabs within strings adding particular underscores
  stepnumber=2,                    % the step between two line-numbers. If it's 1, each line will be numbered
  stringstyle=\color{mymauve},     % string literal style
  tabsize=2,                       % sets default tabsize to 2 spaces
  title=\lstname                   % show the filename of files included with \lstinputlisting; also try caption instead of title
}

% correct bad hyphenation here
\hyphenation{op-tical net-works semi-conduc-tor}

\begin{document}
%
% paper title
% can use linebreaks \\ within to get better formatting as desired
\title{CSP-solver, Assignment 2 IT3105}

\author{Anders Sildnes, Andrej Leitner~\IEEEmembership{students }% <-this % stops a space. jobtitle in memberkj
}% \thanks{Utsendt 2014}}

% The paper headers
\markboth{Report IT3105: Assignment 2}%
{h}

% make the title area
\maketitle

\begin{abstract}
    This text answers assignment 2: combining CSP with best first search.
    The assignment builds from the A* implemented in the previous assignment.
    In this document,
    we explain our implementation of CSP modelled as a search problem.

\end{abstract}

% \begin{IEEEkeywords}
%     Stuff
% \end{IEEEkeywords}
\IEEEPARstart{M}{odeling} problems as CSP has multiple benefits.
The goal is to assign each object a value from a finite domain,
without violating a set of constraints. 
Each node in our graph corresponds to a temporary assignment
to a subset of objects $o \subset O$. If $o$  is complete, that is, every object
has an assigned value, the problem is considered solved.

\section{Initialising the VENT}
The Constraint Network is modelled as a class object called \textit{NET}.
It initialises a set of variable instances (\textit{VI}), each with
a complete domain. The domains are modelled as 1D standard lists with integers.
This can e.g.\ correspond to indexes in a list of 
colors (as used in this assignment), such that the domain denotes
indexes in a (global) list of colors.

The CNET is also responsible for reading (canonical) constraints.
First, it takes in a line of input. The lines of input
are currently just two numbers, although the code should support 
any number \textgreater{} 2. Each number (integer) maps to one variable in the
domain. 

First a general function is created. Currently, it is always assumed
to be a non-equality constraint (to modify this, the input line would need
to be checked for an operator and find a corresponding mapping, which should
be a trivial task). The non-equality constraint does not consider domains
or assignments, it will have general form $x != y$, for any x and y to
be specified later. Here, it is important to map each symbol
to the index of a VI in a state. To do this, only the index of the VI
if stored, and we can lookup the domain from state during invocation. Listing \autoref{lst:constraint}
shows a shorter version of creating a constraint:
\begin{lstinputlisting}[caption="Creation of constraint",language=Python]{example.py}
    \label{lst:constraint}
\end{lstinputlisting}

\begin{description}
    \item[sym\_to\_variable] maps each variable to its position in the
        constraint. E.g. for $x < y$, the position of the variables matters, so the 
        dictionary maps a symbol to a variable.
    \item[c], the \textit{CI} is created once. In the CNET, it is then passed
        on to a list of constraints. This list is indexed by variable;
        such that each variable has a list of constraints it occurs in.
        This means that a constraint with two variables will occur
        twice in the list of constraints. This is to make lookups easier.
\end{description}

The way the constraint reader works now, it should be easy to extend
to more advanced constraints. Apart from the mapping from type of constraint
to code, the rest of the code is written using list comprehensions and Cartesian
product mappings which should be able to test constraints for any number
of variables.

\section{Iterating to a valid solution}
As in the previous assignment, our A-GAC* implementation works with \textit{state} and 
\textit{problems}. The initial state is a copy from CNET, i.e.\ a complete
copy of the CNET's domains and variables. This means that the initial state
is independent to the memory address of the CNET.

From the initial state there is an assumption: it takes a random object $o$, and reduces
the domain to the singleton set. Then, from the CNET, 
it retrieves all the constraints associated with $o$. Each constraint is
then checked using arc-consistency algorithms. The AC-method 
queses constraint-checks for all variables affected by the assumption.
If at any point a constraint forces a domain to become empty, ``False''
is returned and the assumption is left alone.
The revision is done as per Listing 2:

\begin{lstinputlisting}[caption="Checking constraint",language=Python]{satisfy.py}
    \label{lst:enforce}
\end{lstinputlisting}

Each $tup$ is a possible assignment of domain variables. At any point, if the
constraint can be satisfied, we return ``True''

\section{Generality of A*}
The provided source code provides both assignment 1 and assignment 2
as inputs to the same file and code with A*. The only difference is that the CSP and navigation
have different subclasses of \textit{Problem} and \textit{State}. Here the \textit{state} adds
new states by using the method described in the previous paragraph, and assesses
value of states by using the heuristic mentioned in the end of this text.

In other words, to implement the CSP you do need to know anything about A*, you
just need to define a state, a success-full termination and how to progress from
a state.

\section{Generality of A*-GAC}
The code we have provided gives \textit{csp} as its own file. 
It is implemented so that it has its own instances of an 
A*-state, each with their own \textit{VIs} and \textit{CIs}.
The ``Problem'' class is not implemented; this is left as problem specific
(reminder: the Problem class defines successor states and neighbour generation).

To solve a problem in A*-GAC, you implement your own subclass of 
\textit{Problem}. From the CNET you get the initial state. We have set
the generation of neighbours to be specific for each problem. This is because
some problems have different properties. For example, we felt that
propagating from two different random points in the graph was not beneficial.

\section{Generality of CNET}
The root nodes startes by making a deep copy from the original CNET.
Since then, new states are generated by creating a copy from its successor state.
Hence, the CNET is separated from each VI and CI. 

\section{Heuristic}
For the A* algorithm, an index value to each node in the search tree is given.
This is determined by $f(x) = g(x) + h(x)$, where $h(x)$ is a objective value \dots

Our current heuristic is to count the size of all domains to each variable.
The size is taken minus one, such that assigned objects (that has a domain
of size 1) does not count in the value. 

\begin{IEEEbiographynophoto}{Anders Sildnes,}
    4.års student, master informatikk, NTNU.\
\end{IEEEbiographynophoto}
\begin{IEEEbiographynophoto}{Andrej Leitner,}
    4.års student, master informatikk, NTNU.\
\end{IEEEbiographynophoto}

    
% if have a single appendix: %\appendix[Proof of the Zonklar Equations]
% or
%\appendix  % for no appendix heading
% do not use \section anymore after \appendix, only \section*
% is possibly needed
% \appendices
% \section{Proof of the First Zonklar Equation}
% Appendix one text goes here.
%\bibliographystyle{IEEEtran}
%\bibliography{IEEEabrv,../bib/paper}
%
% \begin{thebibliography}{1}
%
% \bibitem{IEEEhowto:kopka}
% H.~Kopka and P.~W. Daly, \emph{A Guide to \LaTeX}, 3rd~ed.\hskip 1em plus
%   0.5em minus 0.4em\relax Harlow, England: Addison-Wesley, 1999.
%
% \end{thebibliography}
% biography section
% 
% If you have an EPS/PDF photo (graphicx package needed) extra braces are
% needed around the contents of the optional argument to biography to prevent
% the LaTeX parser from getting confused when it sees the complicated
% \includegraphics command within an optional argument. (You could create
% your own custom macro containing the \includegraphics command to make things
% simpler here.)
%\begin{biography}[{\includegraphics[width=1in,height=1.25in,clip,keepaspectratio]{mshell}}]{Michael Shell}
% or if you just want to reserve a space for a photo:
\end{document}
