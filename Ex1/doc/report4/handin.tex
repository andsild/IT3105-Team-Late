%% bare_jrnl.tex
%% V1.3
%% 2007/01/11
%% by Michael Shell
%% see http://www.michaelshell.org/
%% for current contact information.
%%
%% This is a skeleton file demonstrating the use of IEEEtran.cls
%% (requires IEEEtran.cls version 1.7 or later) with an IEEE journal paper.
%%
%% Support sites:
%% http://www.michaelshell.org/tex/ieeetran/
%% http://www.ctan.org/tex-archive/macros/latex/contrib/IEEEtran/
%% and
%% http://www.ieee.org/

%%*************************************************************************
%% Legal Notice:
%% This code is offered as-is without any warranty either expressed or
%% implied; without even the implied warranty of MERCHANTABILITY or
%% FITNESS FOR A PARTICULAR PURPOSE! 
%% User assumes all risk.
%% In no event shall IEEE or any contributor to this code be liable for
%% any damages or losses, including, but not limited to, incidental,
%% consequential, or any other damages, resulting from the use or misuse
%% of any information contained here.
%%
%% All comments are the opinions of their respective authors and are not
%% necessarily endorsed by the IEEE.
%%
%% This work is distributed under the LaTeX Project Public License (LPPL)
%% ( http://www.latex-project.org/ ) version 1.3, and may be freely used,
%% distributed and modified. A copy of the LPPL, version 1.3, is included
%% in the base LaTeX documentation of all distributions of LaTeX released
%% 2003/12/01 or later.
%% Retain all contribution notices and credits.
%% ** Modified files should be clearly indicated as such, including  **
%% ** renaming them and changing author support contact information. **
%%
%% File list of work: IEEEtran.cls, IEEEtran_HOWTO.pdf, bare_adv.tex,
%%                    bare_conf.tex, bare_jrnl.tex, bare_jrnl_compsoc.tex
%%*************************************************************************

\documentclass[journal]{IEEEtran}

\newcommand{\subparagraph}{}

\usepackage{amsmath,
            amssymb,
            amsthm,
            atbegshi,
            caption,
            subcaption,
            epigraph,
            etoolbox,
            enumitem,
            fancyhdr,
            geometry,
            graphicx,
            hyperref,
            kpfonts,
            lipsum,
            longtable,
            natbib,
            tabulary,
            thmtools,
            tikz,
            tikzpagenodes,
            titletoc,
            titlesec,
            tocloft,
            url,
            wrapfig
}
\usepackage[utf8]{inputenc}


\begin{document}
%
% paper title
% can use linebreaks \\ within to get better formatting as desired
\title{Solving Puzzles with A*-GAC}

\author{Anders Sildnes, Andrej Leitner~\IEEEmembership{students }% <-this % stops a space. jobtitle in memberkj
}% \thanks{Utsendt 2014}}

% The paper headers
\markboth{Solving Puzzles with A*-GAC}%
{h}

% make the title area
\maketitle

\begin{abstract}
    This text answers assignment 3 and 4: Solving puzzles with our A*-GAC system solver.
\end{abstract}

% \begin{IEEEkeywords}
%     Stuff
% \end{IEEEkeywords}
\IEEEPARstart{C}{SP} solutions defines variables with domains.
A valid solution is one where the domain of all variables has been reduced to the
singleton domain. This occurs by making sure that a:) no constraints are violated
and b) all variables are iterated over, and been locked to a single value in their
domain.

\section*{Generality of A*-GAC Solver}
The solvers for each puzzle is implemented using the same source code as in
the previous project. \textit{Essentially}, we use the exact same code for out 
CNET\footnote{ConstraintNETwork, explained in project 1}. The changes we have made
are:
\begin{enumerate}
    \item Added methods ``addCons(...)'' and ``addLambda(...)'', which can take
        either a lambda, or a string that is parsable as a lambda, and translate
        that into a constraint. 
        Example: addCons([1,2], "A \textless B") adds a constraint for vertex instance
        1 and 2, and maps them onto A, B respectively (the first index maps
        to the first capital letter in the string).
    \item Special case handling for when the domains consist of \textit{sets} rather
        than single numbers, as used in the previous assignment.
\end{enumerate}
Apart from item 1 and 2, this assignement is solved by subclassing the class
``Problem'' and implementing its methods\footnote{in ``astar.py'': triggerStart(), genNeighbour(), destructor(), updateStates()}
and also a parser for the input files.

\section*{Modeling Numberlink}
Both problems could be fully expressed as SAT, so we found both problems to
be NP-hard. We chose to use this representation for the flow puzzle. 
While the benchmark of Michael Spivey 
\small\begin{verbatim}
[http://spivey.oriel.ox.ac.uk
/corner/Programming_competition_results]\end{verbatim}
\normalsize
found that it can be faster to represent point in the graph by its
incoming/outgoing direction of flow (NE, NW, NS, WE, WS \dots), 
we chose SAT because of its simplicity and ease of implementing it in our
model, and also with the knowledge that the input files were no bigger than
$10 \times 10$, and that small differences in performance is not critical to our evaluation.
Also, the SAT formulation meant that we had small domains, which is important
when using our CNET because domains are copied quite often... So therefore, 
even if the winning ``numberlink'' from the benchmark[ibid] modeled 
his project using directions, it might be that it is inefficient in 
our model when using GAC and constraint propagation.

In our model, we represent the board as a 2D-grid of cells, each representing one
vertex instance $v_{i,j}$ for poosition $i,j$ in the cartesian plane.
For each $v$, you can assign $k$ colors, where $k$ is given by
the number of endpoints / 2. 
Each cell $v_{i,j}$ must be a part of the path
\footnote{I will assume the reader is has a mutual understanding of ``path''}
from an endpoint $e_{i,j}$ to another endpoint $e_{i+\alpha,j+\beta}$.
Therefore our constraints are modelled as follows:
\begin{align}
    \forall{i,j}: leastTwoOf( v_{i+1,j},v_{i+1,j},v_{i+1,j},v_{i+1,j}, ) == v_{i,j} \\
    \forall{i,j}: leastOneOf( v_{i+1,j},v_{i+1,j},v_{i+1,j},v_{i+1,j}, ) == e_{i,j}
\end{align}
this can also be thought of as applying a 5-point stencil to the grid
and validating each neighbour value. This constraint only applies to the entir
LOOPS, deffered

\begin{figure}[Hb]
\centering
    \includegraphics[height=2cm,keepaspectratio,width=2.5in]{stencil.jpg}
\caption{Stencil operation}
\label{fig:stencil}
\end{figure}

\subsection{Choice of next cell}
Clearly, random is bad..

\subsection{Choice of heurstic}
We scanned through 


\section*{Modelling Nonograms}
While nonograms certainly also can be expressed as SAT, we found it easier to
follow the model suggested to us in the assignment text. Given a rule
$r$, we generate all possible sequences.

propagates..

\subsection{Choice of heuristic}

\subsection{Choice of next cell}

PICTURES?

\end{document}
