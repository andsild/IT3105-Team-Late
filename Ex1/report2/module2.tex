%% bare_jrnl.tex
%% V1.3
%% 2007/01/11
%% by Michael Shell
%% see http://www.michaelshell.org/
%% for current contact information.
%%
%% This is a skeleton file demonstrating the use of IEEEtran.cls
%% (requires IEEEtran.cls version 1.7 or later) with an IEEE journal paper.
%%
%% Support sites:
%% http://www.michaelshell.org/tex/ieeetran/
%% http://www.ctan.org/tex-archive/macros/latex/contrib/IEEEtran/
%% and
%% http://www.ieee.org/

%%*************************************************************************
%% Legal Notice:
%% This code is offered as-is without any warranty either expressed or
%% implied; without even the implied warranty of MERCHANTABILITY or
%% FITNESS FOR A PARTICULAR PURPOSE! 
%% User assumes all risk.
%% In no event shall IEEE or any contributor to this code be liable for
%% any damages or losses, including, but not limited to, incidental,
%% consequential, or any other damages, resulting from the use or misuse
%% of any information contained here.
%%
%% All comments are the opinions of their respective authors and are not
%% necessarily endorsed by the IEEE.
%%
%% This work is distributed under the LaTeX Project Public License (LPPL)
%% ( http://www.latex-project.org/ ) version 1.3, and may be freely used,
%% distributed and modified. A copy of the LPPL, version 1.3, is included
%% in the base LaTeX documentation of all distributions of LaTeX released
%% 2003/12/01 or later.
%% Retain all contribution notices and credits.
%% ** Modified files should be clearly indicated as such, including  **
%% ** renaming them and changing author support contact information. **
%%
%% File list of work: IEEEtran.cls, IEEEtran_HOWTO.pdf, bare_adv.tex,
%%                    bare_conf.tex, bare_jrnl.tex, bare_jrnl_compsoc.tex
%%*************************************************************************

\documentclass[journal]{IEEEtran}


\usepackage{amsmath,
            amssymb,
            amsthm,
            atbegshi,
            caption,
            subcaption,
            epigraph,
            etoolbox,
            enumitem,
            fancyhdr,
            geometry,
            graphicx,
            hyperref,
            kpfonts,
            lipsum,
            longtable,
            natbib,
            tabulary,
            thmtools,
            tikz,
            tikzpagenodes,
            titletoc,
            titlesec,
            tocloft,
            url,
            wrapfig
}
\usepackage[utf8]{inputenc}

% correct bad hyphenation here
\hyphenation{op-tical net-works semi-conduc-tor}

\begin{document}
%
% paper title
% can use linebreaks \\ within to get better formatting as desired
\title{CSP-solver, Assignment 2 IT3105}

\author{Anders Sildnes, Andrej Leitner~\IEEEmembership{students }% <-this % stops a space. jobtitle in memberkj
}% \thanks{Utsendt 2014}}

% The paper headers
\markboth{Report IT3105: Assignment 2}%
{h}

% make the title area
\maketitle

\begin{abstract}
    This text answers assignment 2
    this is some more text that I can probably fill in somehow.
    Hello my name is alfred.
\end{abstract}

% \begin{IEEEkeywords}
%     Stuff
% \end{IEEEkeywords}
\IEEEPARstart{A*}{} is a local search algorithm. It is similar to DFS and BFS,
except that when choosing a new node, it always chooses a node based on an 
index value. 

\section{Generality of A*}
The source code for A* and the general problem/state instances are given
in one file. The \textit{entire} code for the CSP is given in file \textit{csp}.
We have implemented CSP by subclassing the ``Problem'' and ``State'' class and
implementing the methods given in these superclasses. Apart from specificly
interpreting the input and GUI, the entire CSP logic is in its own ``csp.py'' file.

\section{Generality of CSP}
Figure \autoref{fig:UMLconstraint} shows a constraint class.
It stores a set of 
The class 

\section{Generality of A*-GAC}
Ooooh fuck.

\section{Code Generation}
I use sympy teehee






\begin{IEEEbiographynophoto}{Anders Sildnes,}
    4.års student, master informatikk, NTNU.\
\end{IEEEbiographynophoto}
\begin{IEEEbiographynophoto}{Andrej Leitner,}
    4.års student, master informatikk, NTNU.\
\end{IEEEbiographynophoto}

    
% if have a single appendix: %\appendix[Proof of the Zonklar Equations]
% or
%\appendix  % for no appendix heading
% do not use \section anymore after \appendix, only \section*
% is possibly needed
% \appendices
% \section{Proof of the First Zonklar Equation}
% Appendix one text goes here.
%\bibliographystyle{IEEEtran}
%\bibliography{IEEEabrv,../bib/paper}
%
% \begin{thebibliography}{1}
%
% \bibitem{IEEEhowto:kopka}
% H.~Kopka and P.~W. Daly, \emph{A Guide to \LaTeX}, 3rd~ed.\hskip 1em plus
%   0.5em minus 0.4em\relax Harlow, England: Addison-Wesley, 1999.
%
% \end{thebibliography}
% biography section
% 
% If you have an EPS/PDF photo (graphicx package needed) extra braces are
% needed around the contents of the optional argument to biography to prevent
% the LaTeX parser from getting confused when it sees the complicated
% \includegraphics command within an optional argument. (You could create
% your own custom macro containing the \includegraphics command to make things
% simpler here.)
%\begin{biography}[{\includegraphics[width=1in,height=1.25in,clip,keepaspectratio]{mshell}}]{Michael Shell}
% or if you just want to reserve a space for a photo:
\end{document}
